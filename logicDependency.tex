\documentclass[a4paper]{article}

\usepackage[a4paper,margin=1in]{geometry}

\usepackage[utf8]{inputenc}
\usepackage[T1]{fontenc}

\setlength{\parindent}{0cm}
\setlength{\parskip}{1ex}
\setlength{\itemsep}{3ex}

\setlength{\fboxsep}{0pt}

\begin{document}


\section{Introduction}
Given is a tabular dataset $T$.
Observation tells us that sometimes one can predict some (im-)possible values in one column if the value of an other column is given.

Best example is the \texttt{user-id} and the \texttt{student-id} in a list of all marks for homework in a semester.
Both values usually appear many times but every \texttt{student-id} is matches exactly to one \texttt{user-id}.
We could give a function $f : \texttt{student-id} \to \texttt{user-id}$.
So the \texttt{user-id} is functional dependent to the \texttt{student-id}.
In this example you can we have a special case called bijection as the \texttt{student-id} is functional dependent to the \texttt{user-id} too.
In this case we can easy remove one of these columns and use the function to create the value of this column for a row if needed.

An other example would be the columns \texttt{patient-id} and \texttt{gender} in a table of clinical measurements. 
Value in the column \texttt{patient-id} may or may not repeated.
Usually the gender don't change and we have a function $f : \texttt{patient-id} \to \{ \mbox{female}, \mbox{male} \}$.
As there are usually more than two patients it is obvious that there is no function that maps from \texttt{gender} into \texttt{patient-id}. 

To check if two columns are functional dependent we can do as follow:
\begin{itemize}
\item Build the sets of all used values.\\ (e.g: $A = \{ x : x \mbox{ is in the first column} \}$ and $B = \{ x : x \mbox{ is in the second column} \}$)
\item Consider the relation $\sim_T$ between $A$ and $B$ given by the tables. With $a \sim_T b \Leftrightarrow a$ and $b$ are in the same row in the table for $a \in A$ and $b \in B$.
\item $B$ is functional dependent by $A$ iff for every $a \in A$ there is exactly one $b \in B$.
\end{itemize}

But what is if there is no mapping possible but we know that there are some limitations in $B$ for some values in $A$.
For example that for a table $A = \{ \mbox{female}, \mbox{male} \}$ and $B = \{ \mbox{pregnant}, \mbox{not-pregnant} \}$.
In this case we might have female $\sim_T$ pregnant, female $\sim_T$ not-pregnant and male $\sim_T$ not-pregnant but never \mbox{male $\sim_T$ pregnant}.
Such a case is logically dependency.
We would get a function $F$ that maps $A$ to the powerset of $B$ by $F(a) = \{ b \in B : a \sim_T b \}$. 
In this case $F(\mbox{female}) \{ \mbox{pregnant}, \mbox{not-pregnant} \}$ and $F(\mbox{male}) \{ \mbox{not-pregnant} \}$.

Using this we can say:
\begin{itemize}
\item $B$ is logical dependent by $A$ in table $T$ iff $|F(a)| < |B|$ for at least one $a \in A$.
\item $B$ is functional dependent by $A$ in table $T$ iff $|F(a)| = 1$ for all $a \in A$.
\item There is a bijection between $A$ and $B$ iff $B$ is functional dependent by $A$ and $A$ is functional dependent by $B$.
\end{itemize}

An open question is: can we find a function $Q$ that takes two columns of a table and give us a number that gives us a feeling how good the logical dependency between this two columns is?



\section{Definitions}

\begin{enumerate}
\item The set $[m \ldots n] = \{ m, m + 1, m + 2, \ldots, n \}$ for $m \leq n$.

\item $T$ in a table with $M \geq 2$ columns and $N \geq 1$ rows.

\item Consider two different columns in $T$.
      Let this be $(a_i)_{i \in [1 \ldots N]}$ and $(b_i)_{i \in [1 \ldots N]}$.

\item Let $A$ and $B$ the sets of the present values in this columns.
      \\ $A = \{ a_i : i \in [1 \ldots N] \}$ and $B = \{ b_i : i \in [1 \ldots N] \}$.

\item The table $T$ produces a relation $\sim_T$ between $A$ and $B$ by $a \sim_T b$ if and only if there is a row in $T$ where $a \in A$ and $b \in B$ are in the same row.
      With other words: 
      \[
          \mbox{for all } a \in A, b \in B \mbox{ it holds that }  a \sim_T b \Leftrightarrow \mbox{ there is an } i \in [1..N] \mbox{ with }a_i = a \mbox{ and } b_i = b
      \]

\item The function $F(a) = \{ b \in B : a \sim_T b \}$ gives us all possible $b \in B$ for a given $a \in A$.

%\item Let $P(a,b)$ be the conditional probability that we find a row with $(a,b)$ is in $T$ for a given $a$. E.g:
%\[
%  P(a,b) = \frac{| \{ i \in [1 \ldots N] : a_i = a \mbox{ and } b_i = b\} |}{| \{ i \in [1 \ldots N] : a_i = a \} |}
%\]
\end{enumerate}

\section{The function $Q_T$}

The wishlist:
\begin{itemize}
\item $Q_T(A,B)$ should be zero if $T$ produces a function $f : A \to B$
\item $Q_T(A,B)$ should be one if for every $a \in A$ and $b \in B$ the relation $a \sim_T b$ holds.
\item Every other value of $Q_T(A,B)$ should be between zero and one
\end{itemize}


For a given $a \in A$ we can count how many $b \in B$ holds $a \sim_T b$ by $n = |F(a)| = |\{ b \in B : a \sim_T b \}|$
The more values are excluded the closer we are to a function that takes this $a$ and gives us a $b$.
If there is a function from $A$ to $B$ for this value $a$ then $n = 1$.
By the definition of our Table (not empty) and $A, B$ (= sets of the actual values in the table) we know that $n$ has to be greater or equal to 1.
By subtracting 1 we get: $n' = |F(a)| - 1 \geq 0$

If $a \sim_T b$ for all $b\in B$ for this $a \in A$ then $n' = |B| - 1$.
Assuming $|B| > 1$ we can divide this value and get the function:
\[
  Q_B(a) = \frac{|F(a)| - 1}{|B| - 1}
\]
that maps to the interval $[0, 1]$.
We can sum this up by 
\[
  s = \sum_{a \in A} Q_B(a)
  = \sum_{a \in A} \frac{|F(a)| - 1}{|B| - 1}
  = \frac{1}{|B| - 1} \sum_{a \in A} (|F(a)| - 1)
\]

We can easily see that $s \in [0, |A|]$.

Dividing $|A|$ gives us one solution to our wishlist for $|B| > 1$:
\[
  s' = \frac{s}{|A|}
  = \frac{1}{|A| \cdot (|B| - 1)} \sum_{a \in A} (|F(a)| - 1)
  = \frac{1}{|A| \cdot (|B| - 1)} \sum_{a \in A} (|\{ b \in B : a \sim_T b \}| - 1)
\]
If we put $a$ in the set we can omit the sum and get the easier term:
\[
  s'' = \frac{|\{ (a,b) : a \in A, b \in B \mbox{ and } a \sim_T b \}| - |A|}{|A| \cdot (|B| - 1)}
\]

For the special case that there is only one value in column $B$ (e.g: $B = \{ b \}$) we have the obvious constant function $f(a) = b$.
This leads to our final definition:

\[
\begin{array}{lll}
 Q_T( A, B ) & = s'' & ; if |B| > 1 \\
 Q_T(A, B)   &  = 0  & ; if |B| \leq 1 
\end{array}
\]


\section{Additionally thoughts}
The relation $\sim_T$ builds a bipartite graph.
With $A$ as nodes on one side, $B$ as nodes on the other side and edges when $a \sim_T b$.

We can add a weight to the edges by the conditional probability $P(b|a)$.
Question: Is it possible that this gives some additional insights?


\section{Open questions}
\begin{itemize}
\item how to interpret the non obvious values $> 0$ and $< 1$?
\item how to compare two tables (e.g. real and synthetic data)?
\item What additional information about a table we can produce with this?
\end{itemize}



\end{document}
